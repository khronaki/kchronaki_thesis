\begin{table*}[t]
\begin{center}
\caption{Evaluated benchmarks and relevant characteristics}
\label{tab.apps}
\resizebox{\textwidth}{!}{%
\begin{tabular}{|c|c|c|c|c|c|c|c|c|c|}
\hline
\multirow{3}{*}{\parbox{13mm}{\centering Application}} & 
\multirow{3}{*}{Problem size} & 
\multirow{3}{*}{\parbox{10mm}{\centering \#Tasks}} & 
\multirow{3}{*}{\parbox{17mm}{\centering Avg task CPU cycles (thousands)}} & 
\multicolumn{3}{|c|}{\parbox{22mm}{\centering Per task overheads (CPU cycles)}} & & &\\
\cline{5-7}
& & & & \multirow{2}{*}{\parbox{10mm}{\centering Create}} & \multirow{2}{*}{\parbox{9mm}{\centering All}} & \multirow{2}{*}{\parbox{11mm}{\centering Deps + Sched}} & \multirow{2}{*}{\parbox{15mm}{\centering Measured perf. ratio}} &
{\parbox{10mm}{\centering $r$}} &
\multirow{2}{*}{\parbox{11mm}{\centering Parallel model}} \\
& & & & & & & & & \\ %\hhline{~~~~~~}
\hline

\multirow{2}{*}{\parbox{18mm}{\centering Cholesky factorization}} & 
32K 256 & 357\,762  & 753 & 15221 &  73286 &  58065 &  \multirow{2}{*}{\parbox{9mm}{\centering 3.5}} & 10.34 & \multirow{2}{*}{\parbox{17mm}{\centering dependencies}}\\                                              & 32K 128 & 2829058 & 110 & 17992 &  58820 &  40828 & & 83.74 &\\
%& 32$\times$32 blocks of 512$\times$512 floats & 5984 & & 1\,551\,322 & 104.76 &  238.02 &  194.28  & \\ 
\hline{}
\multirow{2}{*}{\parbox{18mm}{\centering QR factorization}} & 16K 512 & 11\,442 & 518\,570  & 17595 & 63008 &   45413 & \multirow{2}{*}{\parbox{9mm}{\centering 6.8}} & 0.01 &\multirow{2}{*}{\parbox{17mm}{\centering dependencies}}\\
&  16K 128 & 707\,265 & 3\,558 & 21642 & 60777 & 39135 & & 3.11 &\\
\hline
Blackscholes & native & 488\,202 & 348  &   29141  &  85438 &  56297 & 2.3 & 42.87 & data-parallel \\
\hline
Bodytrack & native & 329\,123 & 383 &  9\,505 &  18979 & 9474 & 4.2 & 12.70 & pipeline \\ 
%Heat diffusion & Heat &  &  &  &  &  & \\ 
\hline
Canneal & native & 3\,072\,002 & 67 & 25781 & 50094 &  24313 & 2.0 & 197.01 & unstructured \\
\hline
Dedup & native & 20\,248 & 1\,532 & 1294 & 9647 &  8353 & 2.7 & 0.43 & pipeline \\
\hline 
Ferret & native$\times$2 & 84\,002 & 29\,088 & 38913 & 98457 &  59544 & 3.6 & 0.68 & pipeline \\
\hline
Fluidanimate & native & 128\,502 & 16\,734 & 30210 & 94079 &  64079 & 3.3 & 0.91 & data-parallel \\
\hline
Streamcluster & native & 3\,184\,654 & 161 & 6892 & 13693 &  6801 & 3.5 & 21.91 & data-parallel \\
\hline
\end{tabular}}
\end{center}
\vspace{-0.4cm}
\end{table*}

\subsection{Applications}
%\begin{itemize}
%\item Blackscholes
%\item Cholesky
%\item Canneal
%\item Fluidanimate
%\item QR Factorization
%\item Bodytrack
%\item Streamcluster
%\end{itemize}
Table~\ref{tab.apps} shows the evaluated applications, the input sizes used, and their characteristics. 
All applications are implemented using the OpenMP programming model~\cite{OpenMP4.0:Manual2015}. 
We obtain Cholesky and QR from the BAR repository~\cite{BAR} and we use the implementations of the rest of the benchmarks from the PARSECSs suite~\cite{Chasapis:TACO2016}.
More information about these applications can be found in~\cite{Chasapis:TACO2016} and~\cite{Chronaki:ICS2015}.
As the number of cores in SoCs is increasing, so does the need of available task parallelism~\cite{Sanchez:2010}. 
We choose the input sizes of the applications so that they create enough fine-grained tasks to feed up to 512 cores.
The number of tasks per application and input as well as the average per-task CPU cycles can be found on Table~\ref{tab.apps}.





\subsection{Simulation}
\label{TaskGenX:experimental:simulation}
To evaluate TaskGenX we make use of the TaskSim trace-driven simulator~\cite{AbstrLevels_TACO12,MUSA} as explained in Section~\ref{sec.background.simulation}. 

We evaluate both symmetric and asymmetric systes with the number of cores varying from 8 to 512.
To set the correct performance ratio between big and little cores for the asymetric systems, we measure the sequential execution time of each application on a real ARM big.LITTLE platform when running on a little and on a big core. 
We use the Hardkernel Odroid~XU3 board that includes a Samsung Exynos 5422 chip with ARM big.LITTLE.
The big cores run at 1.6GHz and the little cores at 800MHz.
Table~\ref{tab.apps} shows the measured performance ratio for each case.
The average performance ratio among our 11 workloads is 3.8.
Thus in the specification of the asymmetric systems we use as performance ratio the value 4.

We modify TaskSim so that it features one extra hardware accelerator (per multi-core) responsible for the fast task creation (the RTopt).
Apart from the task duration time, our modified simulator tracks the duration of the runtime overheads.
These overheads include: (a) task creation, (b) dependencies resolution, and (c) scheduling.
The RTopt core is optimized to execute task creation faster than the general purpose cores; 
to determine how much faster a task creation job is executed we use the analysis performed in Section~\ref{sec:hw_req}.

Using Equation~\ref{eq.create}, we compute the $C_{opt}(x)$ for each application according to their average task CPU cycles from Table~\ref{tab.apps} for $x=512$ cores.
$C_{gp}$ is the cost of task creation when it is performed on a general purpose core, namely the \textit{Create} column shown on Table~\ref{tab.apps}.
To have optimal results for each application on systems up to 512 cores, $C_{gp}$ needs to be reduced to $C_{opt}(512)$.
Thus the specialized hardware accelerator needs to perform task creation with a ratio $r = C_{gp}/C_{opt}(512) \times$ faster than a general purpose core.

We compute $r$ for each application shown on Table~\ref{tab.apps}. We observe that for the applications with a large number of per-task CPU cycles and relatively small \textit{Create} cycles (QR512, Dedup, Ferret, Fluidanimate), $r$ is very close to zero, meaning that the task creation cost ($C_{gp}$) is already small enough for optimal task creation without the need of a faster hardware accelerator.
For the rest of the applications, more powerful hardware is needed.
For these applications $r$ ranges from 3$\times$ to 197$\times$.
Comparing $r$ to the measured performance ratio of each application we can see that in most cases accelerating the task creation on a big core would not be sufficient for achieving higher task creation rate.
In our experimental evaluation we accelerate task creation in the RTopt and we use the ratio of 16$\times$ which is a relatively small value within this range that we consider realistic to implement in hardware.
%Contrarily, if RTopt is assigned a task to execute, we assume that it executes it 4$\times$ slower than a general purpose in-order core.
The results obtained show the average results among three different traces for each application-input.