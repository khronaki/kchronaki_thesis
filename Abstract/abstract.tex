\begin{abstract} 
    
    In 2013, U.S. data centres accounted for 2.2\%
    of the country's total electricity consumption, a figure that is projected to increase
    rapidly over the next decade.  A significant proportion of power consumed within a
    data centre is attributed to the servers, and a large percentage of that is wasted as
    workloads compete for shared resources.  Many data centres host interactive workloads
    (e.g., web search or e-commerce), for which it is critical to meet user expectations
    and user experience, called Quality of Service (QoS).  There is also a wish to run
    both interactive and batch workloads on the same infrastructure to increase cluster
    utilisation and reduce operational costs and total energy consumption. Although much
    work has focused on the impacts of shared resource contention, it still remains a
    major problem to maintain QoS for both interactive and batch workloads. The goal of
    this thesis is twofold. First, to investigate how, and to what extent, resource
    contention has an effect on throughput and power of batch workloads via modelling.
    Second, we introduce a scheduling approach to determine on-the-fly the best
    configuration to satisfy the QoS for latency-critical jobs on any architecture.    
\end{abstract}
