%applications
In the evaluation of our contributions we use 13 scientific applications. 
With the prevalence of many-core processors and the increasing relevance of application 
domains that do not belong to the traditional HPC field, comes the need for programs 
representative of current and future parallel workloads. 
The PARSEC benchmark suite~\cite{PARSEC3,Bienia:PhD2011} features state-of-the-art, 
computationally intensive algorithms and very diverse workloads from different areas of computing.
In our experiments, we make use of the original PARSEC codes together with a task-based 
implementation of nine benchmarks of the suite~\cite{Chasapis:TACO2016}. 
Additionaly, we evaluate some representative benchmarks from the BSC Application Repository (BAR)~\cite{BAR}.
These applications are implemented using the OmpSs programming model.

Table~\ref{tab:parsec} describes the benchmarks included in the study along with their respective 
inputs and parallelization strategy. 
We are using native inputs, which are real input sets for native execution, except for \texttt{dedup}, as the entire input file of 672 MB and the intermediate data structures do not fit in the memory system of our platform. 
Instead, we reduce the size of the input file to 351 MB.

%We obtain these applications from the PARSECSs benchmark suite~\cite{Chasapis:TACO2016} as well as from the BSC Application Repository (BAR)~\cite{BAR}.
%These applications are implemented using the OmpSs programming model and/or the pthreads library.
%In our evaluations we either compare our contributions against the default OmpSs runtime or we compare the OmpSs runtime against the application-level (pthreads version) parallelism.
%Table~\ref{tab.apps} shows the applications together with a short description and the parallelization strategy followed in their task-based implementation.
%In the next Chapters, depending on the present evaluation we will update this Table with the appropriate information to facilitate the explanation of the results.


\begin{table*}[h]
	\centering
	\scriptsize
	\caption{Benchmarks used from the PARSEC benchmark suite and their measured performance ratio between big and little cores}
    %\vspace{-0.2cm}
	\setlength{\tabcolsep}{3pt}
	\begin{tabular}{|p{2cm}|p{5.7cm}|p{4.5cm}|c|}
	\hline
	\textbf{Benchmark} & \multicolumn{1}{|c|}{\textbf{Description}} & \multicolumn{1}{|c|}{\textbf{Input}} & \textbf{Parallelization} \\
	\hline \hline
	Blackscholes & Calculates the prices of a portfolio analytically with the Black-Scholes partial differential equation. & 10,000,000 options & data-parallel \\ \hline
	Bodytrack & Computer vision application which tracks a 3D pose of a marker-less human body with multiple cameras through an image sequence. & 4 cameras, 261 frames, 4,000 particles, 5 annealing layers & pipeline\\ \hline
	Canneal & Simulated cache-aware annealing to optimize routing cost of a chip design. & 2.5 million elements, 6,000 steps & unstructured\\ \hline
	Cholesky Factorization & Dense matrix operation that is used for solving linear equations in linear least square systems. & \kc{multiple} & dependencies\\ \hline
	Dedup & Compresses a data stream with a combination of global compression and local compression in order to achieve high compression ratios. & 351 MB data & pipeline\\ \hline
	Facesim & Takes a model of a human face and a time sequence of muscle activation and computes a visually realistic animation of the modeled face. & 100 frames, 372,126 tetrahedra & data-parallel\\ \hline
	Ferret & Content-based similarity search of feature-rich data (audio, images, video, etc.) & 3,500 queries, 59,695 images database, find top 50 images & pipeline\\ \hline
	Fluidanimate & Extended Smoothed Particle Hydrodynamics method to simulate an 
incompressible fluid for interactive animations. & 500 frames, 500,000 particles & 
data-parallel\\ \hline
	Heat diffusion & Computes the heat distribution on a matrix from \textit{x} heat sources using the Gaus-Seidel method. & 16$\times$16 blocks of 512$\times$512 doubles & data-parallel\\ \hline
	Integral Histogram & A method to compute a cumulative histogram for each pixel of an image represented as a Cartesian data space in constant time. & 8$\times$8 blocks of 12$\times$512 floats & dependencies \\ \hline
	QR Factorization & A linear algebra algorithm that is used to solve the linear least squares problem \cite{QR}.& \kc{multiple} & dependencies \\ \hline
	Streamcluster & Solves the online clustering problem. & 200K points per block, 5 block & 
data-parallel\\ \hline
	Swaptions & Intel RMS workload; uses the Heath-Jarrow-Morton framework to price a portfolio of swaptions. & 128 swaptions, 1 million  simulations & data-parallel\\ \hline
%	vips & VASARI Image Processing System (VIPS), which includes fundamental image processing operations. & 18,000$\times$18,000 pixels & 127,957\\ \hline
%	x264 & H.264/AVC (Advanced Video Coding) video encoder. & 512 frames, 1,920$\times$1,080 pixels & 29,329\\ \hline 
	\end{tabular}
	\label{tab:parsec}
	%\vspace{-0.3cm}
\end{table*}
