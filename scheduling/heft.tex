\section{Dynamic Heterogeneous Earliest Finish Time Scheduler (dHEFT)}
\label{sec.scheduling.heft}
The Heterogeneous Earliest Finish Time (HEFT) algorithm~\cite{HEFT} is a well-known state-of-the-art static scheduling algorithm for asymmetric systems.
HEFT consists of two compile-time phases that use profiling information: the \textit{task prioritizing phase} and the \textit{processor selection phase}.
In the first phase, the algorithm assigns priorities to the tasks based on their \textit{upward rank}, that is, the length of the critical path from a given task to the exit task including task computation and communication costs~\cite{HEFT}.
When task prioritizing is done, the tasks are sorted according to their priorities.
In the \textit{processor selection phase} the algorithm searches for each task the appropriate processor to execute it.
By keeping communication and computation costs, HEFT assigns each task to the processor that will finish its execution at the earliest possible time.
Topcuoglu et al. \cite{HEFT} present their results based on evaluation on synthetic TDGs and assume known task execution and communication times at compile time.
The scheduling is static, so all the decisions are taken before execution.

In this paper, since the evaluation consists of running real applications with unknown task costs, the best way to compare HEFT to our proposal is by using a dynamic version of HEFT algorithm (dHEFT).
The dHEFT is implemented in the OmpSs programming model and is based on the implementation used in the evaluation of CATS~\cite{Chronaki:ICS2015}.
This version assumes two different types of cores (fast and slow) and keeps records of the task costs in each core.
DHEFT discovers the task costs at runtime, computes the mean cost of each tt-is for each core type and then finds the core that will finish the task at the earliest possible time.

To find the earliest possible executor, dHEFT maintains
one list per core (wlist) including the ready tasks waiting
to be executed by that core. 
When a task becomes ready, dHEFT first inserts it in the ordered ready queue; then the task with the highest upward rank is selected and dHEFT checks if there are execution time records for this task. 
If the number of records is sufficient (we require a minimum of three records)
then the estimated cost of the task is considered stable. 
Using that estimated execution time, the task
is scheduled to the earliest executor by consulting the wlist
of all cores. If the number of records is not sufficient
for one of the core types, then the task is scheduled to the
earliest executor of this core type to get another record of
that task-type and core-type execution time. In all cases,
dHEFT updates the history of records on every task execution to adapt for phase changes in the application. \footnote{The time complexity of the task submission step is \textit{O($nN$)} and the task-to-core assignment is \textit{O($n$)}, where \textit{$n$} is the number of tasks and \textit{$N$} is the number of cores.}

The initial dHEFT version presented in previous work~\cite{Chronaki:ICS2015} lacks the \textit{task prioritizing phase} of the original HEFT algorithm.
This paper, uses an improved version of dHEFT that adds this functionality by prioritizing tasks according to their \textit{upward rank}.
The implementation of this is similar to the CPATH prioritization step.
%to the tasks whenever an undiscovered tt-is has finished its execution.
%The TDG is traversed from the top to the bottom and, using task execution times, the upward-rank-based priorities are assigned to tasks.
When the prioritized tasks become ready, they are inserted in a sorted ready queue in decreasing order of their priorities.
The algorithm then accesses the tasks in the order of their priorities to find the earliest executor for each of them.
